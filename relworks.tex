\documentclass[sigplan,nonacm]{acmart}
\settopmatter{printfolios=true}
\usepackage{graphicx} % Required for inserting images
\graphicspath{ {./images/} } %images subfolder
\usepackage[normalem]{ulem}

\title{Related Works}
\author{Mark Madler}
\begin{document}
\maketitle
%All are related to some degree -- so they are organized as follows:

%                       REPLICATED KVS
%------------------------------------------------------------------------------------

\section{KVS over RDMA}

    \subsection{Kite: efficient and available release consistency for the datacenter}
    This is the real entry paper. This is a replicated KVS over RDMA with a 
    "Linearizable variant of" \textbf{Release Consistency}. Some mention of Release Consistency
    as a "one sided barrier." Compares against Zookeeper and Derecho. Uses some 
    monotonically increasing clocks to track versions.\cite{Gavrielatos-PPoPP-2020} 

    \subsection{FaRM: Fast Remote Memory}
    Super similar to the entry paper in that is is a KVS for RDMA, but this one is I beleive either disaggregated or not. 
    Farm could really be classified as a KVS or even a protocol. This design always replicates state info. 
    Provides \textbf{strict serializability}.\cite{Dragojevic-NSDI-2014}

    \subsection{FaRMv2: Fast General Distributed Transactions with Opacity}
    Just like FaRM but with opacity. Also providing \textbf{strict serializability}.\cite{Shamis-SIGMOD-2019}

    \subsection{HERD: Using RDMA efficiently for key-value services}
    Herd. Fast because it uses a weird combination of UD and message things.\cite {Kalia-SIGCOMM-2014}

    \subsection{Sherman: A Write-Optimized Distributed B+Tree Index on Disaggregated Memory}
    This system is disaggregated. Node-granularity. "consistent". Optimizations are established
    to target mainly chained atomic operations and write-amplification. \cite{Wang-SIGMOD-2022}

    \subsection{Using One-Sided RDMA Reads to Build a Fast, CPU-Efficient Key-Value Store (Pilaf) }
    Linearizable data store.  Self-correcting. Note that this paper has a great 
    analysis of IPoIB as well as RDMA vs verbs latency and thrup. \cite{Mitchell-ATC-2013}

    \subsection{Scythe: A Low-Latency RDMA-enabled Distributed Transaction System for Disaggregate Memory }
    KVS again. Seels to optimize concurrency control, timestamping, bandwidth. Uses Timestamp Oracle (TSO)
    hot-aware concurrecy control. Allows RPC. \cite{Lu-ACMtrans-2024}

    \subsection{HCL: Distributing Parallel Data Structures in Extreme Scales}
    Strictly serializabile KVS. HCL stands for Hermes Containder Library. \cite{Devarajan-CLUSTER-2020}

    \subsection{BCL: A Cross-Platform Distributed Data Structures Library}
    Confusingly uses either MPI, OpenSHMEM, GASNET-EX, and UPC++ as backends. Must be slow. \cite{Brock-ICPP-2019}
    
    \subsection{Rolex}
    I don't think this one really works. Uses learned indexes. splits compute and memory. \cite{Li-FAST-2023}

    \subsection{Exploiting Hybrid Index Scheme for RDMA-based Key-Value Stores*** (Hstore)}
    Seeks to solve issue of range-lookups. Compares against Sherman and Clover( LOOK INTO CLOVER). 
    This does indeed outperform sherman on YCSB. Maybe problematic for LOCO. claims 54\% improvement 
    over sherman.\cite{Han-SYSTOR-2023}

    \subsection{Disaggregating Persistent Memory and Controlling Them Remotely: An Exploration of Passive Disaggregated Key-Value Stores}
    "Clover". Workes on Persistent memory. A bit older, not worth comparing against unless looking into PM. also maybe faster 
    than Sherman but definitely not the fastest.\cite{Tsai-USENIX-2020}

    \subsection{AStore: Uniformed Adaptive Learned Index and Cache for RDMA-Enabled Key-Value Store}
    Supposedly outperforms \textbf{Sherman} by A LOT. ALso compares against "xstore" "Rolex" and "hybrid (not to be confused
    with the above hybrid index scheme which also independetly claims to be faster than Sherman)". 
    The results also suggest that Rolex outperforms \textbf{Sherman}. claims up to 91\% improvement over \textbf{Sherman}\cite{Qiao-IEEEtrans-2024}

    \subsection{Maintaining Cache Consistency in RDMA-based Distributed Key-Value storage System}
    note really worth considering it seems. This is indeed a KVS but only seems to compare against scale store. \cite{Hou-DSIT-2024}
%                       DSM OVER RDMA
%-----------------------------------------------------------------------------------
\section {DSM systems}

    \subsection{Scaling out NUMA-Aware Applications with RDMA-Based Distributed Shared Memory: MAGI}
    Page-based DSM. \cite{Hong-JCST-2019}

    \subsection{Efficient Distributed Memory Management with RDMA and Caching }
    cache-line granularity DSM.\cite{Cai-VLDB-2018}

    \subsection{Distributed Shared Object Memory}
    object based granularity, release consistency... too old for RDMA\cite{Guedes-WWOSIII-1993}

    \subsection{Gengar: An RDMA-based Distributed Hybrid Memory Pool}
    This is object based dsm over rdma but with non-volatile memory as well using Intel Optane. Seems to 
    also use this lease assignment idea like in \cite{Endo-IPDRM-2020} but is not page based.\cite{Duan-ICDCS-2021}

    \subsection{TreadMarks: shared memory computing on networks of workstations}
    Was implemented over IP, lazy release consistency I think. Not sure of granularity yet.\cite{Amza-Usenix-1994}

    \subsection{LITE Kernel RDMA Support for Datacenter Applications}
    This is page based DSM using the kernel. \cite{Tsai-SOSP-2017}

    \subsection{MENPS: A Decentralized Distributed Shared Memory Exploiting RDMA}
        \begin{itemize}
            \item Page based DSM
            \item Special Diff merging and page sharing
            \item Combine write notices and logical leases (what is that?)\cite{Endo-IPDRM-2020}
        \end{itemize}

    \subsection{Argo DSM}
    Page-based DSM again but directory coherence. This was maybe the first RDMA-based DSM paper, 
    at least thats what the authors allude to.\cite{Kaxiras-HPDC-2015}

    \subsection {GiantVM: A Novel Distributed Hypervisor for Resource Aggregation with DSM-aware Optimizations}
    Page-based DSM again but also works over TCP and RDMA\cite{Jia-ACO-2022}

    \subsection{Scalable RDMA performance in PGAS languages}
    This paper is for PGAS languages. Has an address hash table similar to LOCO for remote lookups.\cite{Farreras-IPDPS-2009}

    \subsection{Misc PGAS languages probably}
%-----------------------------------------------------------------------------------------
%                       PROTOCOLS OVER RDMA FOR CONSISTENCY
%----------------------------------------------------------------------------------------
\section{Protocols over RDMA for Consistency}

    \subsection{Notes on PGAS and "protocols"}
    It seems like there are not agreed upon semantics on what is a protocol. MPI seems like a protocol but is it? PGAS is a 
    memory model.

    \subsection {Odyssey: The Impact of Modern Hardware on Strongly-Consistent Replication Protocols}
    This paper is a summary of protocols used for RDMA communication. These protocols were used to 
    enforce consistency and were tested with a series of KVSs. This paper is related to Kite (same authors) 
    and Kite is one of the KVSs tested.\cite{Gavrielatos-EuroSys-2021}

    \subsection {Hermes: A Fast, Fault-Tolerant and Linearizable Replication Protocol}
    This paper \cite{Katsarakis-ASPLOS-2020} is one of the Protocols tested by the above paper Odyssey \cite{Gavrielatos-EuroSys-2021}. This 
    protocol guarantees linearizablity and is designed to work on replicated store systems.

    \subsection {Hamband: RDMA Replicated Data Types}
    This paper \cite{Houshmand-PLDI-2022} designed new RDMA data types that are replicated across nodes. This paper 
    is sort of a protocol paper as it implements this protocol to keep replicated data through either relaxed or 'strong consistency'.

    \subsection{Evaluation of RDMA opportunities in an Object-Oriented DSM}
    Interesting result is that it proves that invalidation protocols are better suited 
    for distributed systems. \cite{Veldema-LCPC-2007}
%----------------------------------------------------------------------------------------
%                       AWESOME TABLE
%----------------------------------------------------------------------------------------
\section{table i found}
\includegraphics[width=0.5\textwidth]{Table_1_A_Survey_of_Storage_Systems_in_the_RDMA_Era}
\cite{Ma-PDS-2022}
%-----------------------------------------------------------------------------------------
%                       LOOSELY RELATED
%-----------------------------------------------------------------------------------------
%Loosely Related / Line of Work 
\section{Loosely Related but Evaluated}
    \subsection{CoRM: Compactable Remote Memory over RDMA}
    page based I think (re-read this)\cite{Taranov-ICMD-2021}

    \subsection{Rcmp: Reconstructing RDMA-Based Memory Disaggregation via CXL}
    page based and uses CXL, not comparable\cite{Wang-ACO-2024}
%-----------------------------------------------------------------------------------------


\bibliographystyle{acm}
\bibliography{relworks}
\end{document}